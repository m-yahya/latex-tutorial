\documentclass[12pt, letter]{article}
\usepackage[utf8]{inputenc}

\usepackage{enumitem}

\setlength\parindent{0pt}

\begin{document}

\listoftables
\newpage

\section{Create Table}
A simple table can be created using the `tabular` environment which is the default method to create tables in LaTeX. The \ref{table1} is an example of  a very simple table.

\begin{table}[h!]
  \centering
  \caption{A Simple Table with Caption and Label}
  \label{table1}
  \begin{tabular}{ c c c }
    \hline
    row1-col1 & row1-col1 & row1-col1 \\
    \hline
    row1-col1 & row1-col2 & row1-col3 \\
    row2-col1 & row2-col2 & row2-col3 \\
    row3-col1 & row3-col2 & row3-col3 \\
    \hline
  \end{tabular}
\end{table}

\section{Create Lists}
The following is an un-ordered list of items.

\begin{itemize}
  \item The un-unmbered item in the list
  \item Another un-numbered item in the lists
  \item One more un-numbered item in the list
\end{itemize}

The following is an example of numbered list.

\begin{enumerate}
  \item The first item in the list
  \item The second item in the list
  \item The third item in the list
\end{enumerate}

Its also possible to create nested lists in \LaTeX.

\begin{enumerate}
  \item The first item in the list
  \item The second item in the list
        \begin{enumerate}
          \item sub item a
          \item sub item b
          \item sub item c
        \end{enumerate}
   \item The third item in the list
\end{enumerate}

\subsection{Customizing List Style}

By defualt, LaTeX uses black dot for bullet points in un-numbered list. This default behaviour can be changed to bold dash, dash and asterik in the following way:

\begin{itemize}
	\item[--] Dash item
    \item[$-$] Bold Dash item
    \item[$\ast$] Asterisk item
\end{itemize}

The numbering style for ordered list can be changed to \textit{roman}, \textit{arabic} and \textit{alph} as follows:

\begin{enumerate}[label=\roman*]
  \item The first item with Roman numbers
  \item The second item with Roman numbers
  \item The third item with Roman numbers
\end{enumerate}

\begin{enumerate}[label=(\arabic*)]
  \item The first item with Arabic numbers
  \item The second item with Arabic numbers
   \item The third item with Arabic numbers
\end{enumerate}

\begin{enumerate}[label=(\alph*)]
  \item The first item with Alphabetical numbers
  \item The second item with Alphabetical numbers
   \item The third item with Alphabetical numbers
\end{enumerate}

\end{document}